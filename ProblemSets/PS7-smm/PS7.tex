\documentclass[12pt,english]{article}
\usepackage{mathptmx}
\usepackage[utf8]{inputenc}
\usepackage{babel}
\usepackage{geometry}
\usepackage{color}
\usepackage[dvipsnames]{xcolor}
\definecolor{byublue}     {RGB}{0.  ,30. ,76. }
\definecolor{darkblue}    {RGB}{0.  ,0.  ,139.}
\definecolor{dukeblue}    {RGB}{0.  ,0.  ,156.}
\geometry{verbose,tmargin=1in,bmargin=1in,lmargin=1in,rmargin=1in}
\usepackage{amsmath}
\usepackage[authoryear]{natbib}
\usepackage{ulem}
\usepackage{minted}
\usepackage{mathtools}
\definecolor{bg}{rgb}{0.95,0.95,0.95}
\usepackage[backref=page]{hyperref}                                              % Always add hyperref (almost) last
\hypersetup{unicode=true,bookmarksnumbered=true,bookmarksopen=true,bookmarksopenlevel=3,
 breaklinks=true,pdfborder={0 0 0},colorlinks,citecolor=darkblue,filecolor=darkblue,linkcolor=darkblue,urlcolor=darkblue,pagebackref=true}
\usepackage[all]{hypcap}                                            % Links point to top of image, builds on hyperref
\usepackage{breakurl}

\begin{document}

\title{Problem Set 7}
\author{ECON 6343: Econometrics III\\
Prof. Tyler Ransom\\
University of Oklahoma}
\date{}%{Due: October 15, 9:00 AM}

\maketitle
Directions: Answer all questions. Each student must turn in their own copy, but you may work in groups. You are encouraged to use any and all Artificial Intelligence resources available to you to complete this problem set. Clearly label all answers. Show all of your code. Turn in jl-file(s), output files and writeup via GitHub. Your writeup may simply consist of comments in jl-file(s). If applicable, put the names of all group members at the top of your writeup or jl-file.

You may need to install and load the following package:
\begin{itemize}
 \item[~] \texttt{SMM}
\end{itemize}

You will need to load the following previously installed packages:
\begin{itemize}
    \item[~] \texttt{Optim} 
    \item[~] \texttt{HTTP} 
    \item[~] \texttt{GLM} 
    \item[~] \texttt{LinearAlgebra} 
    \item[~] \texttt{Random} 
    \item[~] \texttt{Statistics} 
    \item[~] \texttt{DataFrames} 
    \item[~] \texttt{DataFramesMeta} 
    \item[~] \texttt{CSV} 
\end{itemize}
\pagebreak
In this problem set, we will practice estimating models by Generalized Method of Moments (GMM) and Simulated Method of Moments (SMM).

\begin{enumerate}
\item Estimate the linear regression model from Question 2 of Problem Set 2 by GMM. Write down the moment function as in slide \#8 of the Lecture 9 slide deck and use \texttt{Optim} for estimation. Use the $N\times N$ Identity matrix as your weighting matrix. Check your answer using the closed-form matrix formula for the OLS estimator.

\item Estimate the multinomial logit model from Question 5 of Problem Set 2 by the following means:
    \begin{enumerate}
    \item Maximum likelihood (i.e. re-run your code [or mine] from Question 5 of Problem Set 2)
    \item GMM with the MLE estimates as starting values. Your $g$ object should be a vector of dimension $N\times J$ where $N$ is the number of rows of the $X$ matrix and $J$ is the dimension of the choice set. Each element, $g$ should equal $d - P$, where $d$ and $P$ are ``stacked'' vectors of dimension $N\times J$
    \item GMM with random starting values
    \end{enumerate}
    Compare your estimates from part (b) and (c). Is the objective function globally concave?

\item Simulate a data set from a multinomial logit model, and then estimate its parameter values and verify that the estimates are close to the parameter values you set. That is, for a given sample size $N$, choice set dimension $J$ and parameter vector $\beta$, write a function that outputs data $X$ and $Y$. I will let you choose $N$, $J$, $\beta$ and the number of covariates in $X$ ($K$), but $J$ should be larger than 2 and $K$ should be larger than 1. If you haven't done this before, you may want to follow these steps:
    \begin{enumerate}
    \item Generate $X$ using a random number generator---\texttt{rand()} or \texttt{randn()}.
    \item Set values for $\beta$ such that conformability with $X$ and $J$ is satisfied
    \item Generate the $N\times J$ matrix of choice probabilities $P$
    \item Draw the preference shocks $\epsilon$ as a $N\times 1$ vector of $U[0,1]$ random numbers
    \item Generate $Y$ as follows:
        \begin{itemize}
        \item Initialize $Y$ as an $N\times 1$ vector of 0s
        \item Update $Y_i = \sum_{j=1}^J 1\left[\left\{\sum_{k=j}^J  P_{ik}\right\} > 
        \epsilon_i\right]$
        \end{itemize}
    \item An alternative way to generate choices would be to draw a $N\times J$ matrix of $\epsilon$'s from a T1EV distribution. This distribution is already defined in the \texttt{Distributions} package. Then $Y_i = \arg \max_{j} X_i \beta_j + \epsilon_{ij}$. I'll show you an example of how to do that in the solutions code for this problem set.
    \end{enumerate}
\item \sout{Use \texttt{SMM.jl} to run the example code on slide \#21 of the Lecture 9 slide deck.}

\item Re-estimate multinomial logit model from Question 2 using SMM. It will be helpful to use the code example from slide \#18 of the Lecture 9 slide deck. You will also want to make use of your code from Question 3 to do this. (You can't do this question without using the code from Question 3.)

\item As with previous problem sets, please wrap all of your code in a function so that you don't evaluate things in the global scope.


\item As with previous problem sets, have an AI write unit tests for each of the functions you've created (or components of each) and run them to verify that they work as expected. Best practice is to provide unit tests in a separate script that first reads in the source code before running the tests.

\end{enumerate}
\end{document}

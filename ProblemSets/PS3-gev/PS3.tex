\documentclass[12pt,english]{article}
\usepackage{mathptmx}
\usepackage[utf8]{inputenc}
\usepackage{babel}
\usepackage{geometry}
\usepackage{xcolor}
\geometry{verbose,tmargin=1in,bmargin=1in,lmargin=1in,rmargin=1in}
\usepackage{amsmath}
\usepackage[authoryear]{natbib}
\usepackage{minted}
\definecolor{bg}{rgb}{0.95,0.95,0.95}
\usepackage[unicode=true,pdfusetitle,
 bookmarks=true,bookmarksnumbered=false,bookmarksopen=false,
 breaklinks=false,pdfborder={0 0 0},pdfborderstyle={},backref=false,colorlinks=false]
 {hyperref}
\usepackage{breakurl}

\begin{document}

\title{Problem Set 3}
\author{ECON 6343: Econometrics III\\
Prof. Tyler Ransom\\
University of Oklahoma}
\date{}%{Due: September 17, 9:00 AM}

\maketitle
Directions: Answer all questions. Each student must turn in their own copy, but you may work in groups. You are encouraged to use any and all Artificial Intelligence resources available to you to complete this problem set. Clearly label all answers. Show all of your code. Turn in jl-file(s), output files and writeup via GitHub. Your writeup may simply consist of comments in jl-file(s). If applicable, put the names of all group members at the top of your writeup or jl-file.

You will need to load the following packages:
\begin{itemize}
    \item[~] \texttt{Optim} 
    \item[~] \texttt{HTTP} 
    \item[~] \texttt{GLM} 
    \item[~] \texttt{LinearAlgebra} 
    \item[~] \texttt{Random} 
    \item[~] \texttt{Statistics} 
    \item[~] \texttt{DataFrames} 
    \item[~] \texttt{CSV} 
    \item[~] \texttt{FreqTables}
\end{itemize}

On Github there is a file called \texttt{PS2\_starter.jl} that has the code blocks below already created.

\begin{enumerate}
\item Estimate a multinomial logit (with alternative-specific covariates $Z$) on the following data set:

\begin{minted}[bgcolor=bg]{julia}
using DataFrames
using CSV
using HTTP
url = "https://raw.githubusercontent.com/OU-PhD-Econometrics/fall-2024/
master/ProblemSets/PS3-gev/nlsw88w.csv"
df = CSV.read(HTTP.get(url).body, DataFrame)
X = [df.age df.white df.collgrad]
Z = hcat(df.elnwage1, df.elnwage2, df.elnwage3, df.elnwage4, 
         df.elnwage5, df.elnwage6, df.elnwage7, df.elnwage8)
y = df.occupation
\end{minted}

The choice set represents possible occupations and is structured  as follows.

\begin{enumerate}
    \item[1] Professional/Technical 
    \item[2] Managers/Administrators
    \item[3] Sales                  
    \item[4] Clerical/Unskilled     
    \item[5] Craftsmen              
    \item[6] Operatives             
    \item[7] Transport              
    \item[8] Other                  
\end{enumerate}

\textbf{Hints:} 
\begin{itemize}
    \item Index the parameter vector so that the coefficient on $Z$ is the last element and the coefficients on $X$ are the first set of elements.
    \item You will need to difference the $Z$'s in your likelihood function. 
    \item Normalize $\beta_J = 0$
    \item The formula for the choice probabilities will thus be
    \begin{align*}
        P_{ij} &= \begin{cases} \frac{\exp\left(X_{i}\beta_j + \gamma(Z_{ij}-Z_{iJ})\right)}{1+\sum_{k=1}^{J-1}\exp\left(X_{i}\beta_k + \gamma(Z_{ik}-Z_{iJ})\right)} ,& j=1,\ldots,J-1\\
         \frac{1}{1+\sum_{k=1}^{J-1}\exp\left(X_{i}\beta_k + \gamma(Z_{ik}-Z_{iJ})\right)} ,& j=J
         \end{cases}
    \end{align*}
\end{itemize}

\item Interpret the estimated coefficient $\hat{\gamma}$.

\item Estimate a nested logit with the following nesting structure:
\begin{itemize}
    \item White collar occupations (indexed by $WC$)
    \begin{itemize}
    \item[1] Professional/Technical 
    \item[2] Managers/Administrators
    \item[3] Sales 
    \end{itemize}
    \item Blue collar occupations (indexed by $BC$)
    \begin{itemize}
    \item[4] Clerical/Unskilled     
    \item[5] Craftsmen              
    \item[6] Operatives             
    \item[7] Transport              
    \end{itemize}
    \item Other occupations (indexed by $\text{Other}$)
    \begin{itemize}
    \item[8] Other 
    \end{itemize}
\end{itemize}

Specify the parameters such that there are only nest-level (rather than alternative-level) coefficients. That is, estimate a model with the following parameters:
\begin{itemize}
    \item $\beta_{WC}$
    \item $\beta_{BC}$
    \item $\lambda_{WC}$
    \item $\lambda_{BC}$
    \item $\gamma$
    \item $\beta_{\text{Other}}$ is normalized to 0
    \item The formula for the choice probabilities will thus be
    \begin{align*}
        P_{ij} &= \begin{cases} \frac{\exp\left(\frac{X_{i}\beta_{WC}+\gamma(Z_{ij}-Z_{iJ})}{\lambda_{WC}}\right)\left[\sum_{\ell\in WC}\exp\left(\frac{X_{i}\beta_{WC}+\gamma(Z_{i\ell}-Z_{iJ})}{\lambda_{WC}}\right)\right]^{\lambda_{WC}-1}}{1+\left[\sum_{k\in WC}\exp\left(\frac{X_{i}\beta_{WC}+\gamma(Z_{ik}-Z_{iJ})}{\lambda_{WC}}\right)\right]^{\lambda_{WC}}+\left[\sum_{m\in BC}\exp\left(\frac{X_{i}\beta_{BC}+\gamma(Z_{im}-Z_{iJ})}{\lambda_{BC}}\right)\right]^{\lambda_{BC}}} ,& j\in WC\\
        \frac{\exp\left(\frac{X_{i}\beta_{BC}+\gamma(Z_{ij}-Z_{iJ})}{\lambda_{BC}}\right)\left[\sum_{\ell\in BC}\exp\left(\frac{X_{i}\beta_{BC}+\gamma(Z_{i\ell}-Z_{iJ})}{\lambda_{BC}}\right)\right]^{\lambda_{BC}-1}}{1+\left[\sum_{k\in WC}\exp\left(\frac{X_{i}\beta_{WC}+\gamma(Z_{ik}-Z_{iJ})}{\lambda_{WC}}\right)\right]^{\lambda_{WC}}+\left[\sum_{m\in BC}\exp\left(\frac{X_{i}\beta_{BC}+\gamma(Z_{im}-Z_{iJ})}{\lambda_{BC}}\right)\right]^{\lambda_{BC}}} ,& j\in BC\\
         \frac{1}{1+\left[\sum_{k\in WC}\exp\left(\frac{X_{i}\beta_{WC}+\gamma(Z_{ik}-Z_{iJ})}{\lambda_{WC}}\right)\right]^{\lambda_{WC}}+\left[\sum_{m\in BC}\exp\left(\frac{X_{i}\beta_{BC}+\gamma(Z_{im}-Z_{iJ})}{\lambda_{BC}}\right)\right]^{\lambda_{BC}}} ,& j=J
         \end{cases}
    \end{align*}

\end{itemize}


\item Wrap all of your code above into a function and then call that function at the very bottom of your script. Make sure you add \texttt{println()} statements after obtaining each set of estimates so that you can read them.

<<<<<<< HEAD
\item Have an AI write unit tests for each of the functions you've created (or components of each) and run them to verify that they work as expected. Best practice is to provide unit tests in a separate script that first reads in the source code before running the tests.
=======
\item Have an AI write unit tests for each of the functions you've created (or components of each) and run them to verify that they work as expected. Best practice is to provide unit tests in a separate script that first reads in the source code before running the tests. Thus, I would like you to turn in three files:
    \begin{itemize}
        \item \texttt{PS3\_LastName\_source.jl}
        \item \texttt{PS3\_LastName\_script.jl}
        \item \texttt{PS3\_LastName\_tests.jl}
    \end{itemize}
>>>>>>> 8fb07108de6ec9d237ad47c093d7b8fe534debde

\end{enumerate}
\end{document}
